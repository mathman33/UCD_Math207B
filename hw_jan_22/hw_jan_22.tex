\documentclass[12pt]{article}

\usepackage{amssymb, amsmath, amsfonts}
\usepackage{moreverb}
\usepackage{graphicx}
\usepackage{enumerate}
\usepackage[margin=0.75in]{geometry}
\usepackage{graphics}
\usepackage{color}
\usepackage{array}
\usepackage{float}
\usepackage{hyperref}
\usepackage{textcomp}
\usepackage{bbold}
\usepackage{alltt}
\usepackage{physics}
\usepackage{mathtools}
\usepackage{amsthm}
\usepackage{tikz}
\usetikzlibrary{positioning}
\usetikzlibrary{arrows}
\usepackage{pgfplots}
\usepackage{bigints}
\allowdisplaybreaks
\pgfplotsset{compat=1.12}

\theoremstyle{plain}
\newtheorem*{theorem*}{Theorem}
\newtheorem{theorem}{Theorem}
\newtheorem*{lemma*}{Lemma}
\newtheorem{lemma}{Lemma}

\newenvironment{definition}[1][Definition]{\begin{trivlist}
\item[\hskip \labelsep {\bfseries #1}]}{\end{trivlist}}

\title{\bf HW \#1}
\author{\bf Sam Fleischer}
\date{\bf January 22, 2015}

\pgfplotsset{compat=1.12}

\begin{document}
\textbf{MATH 207B \hfill Applied Math \ \ \hfill Winter 2016\ \ \ }

{\let\newpage\relax\maketitle}

\section*{Problem 1}
\textit{Define $f\ |\ \mathbb{R}^2 \rightarrow \mathbb{R}$ by $f(0, 0) = 0$ and}
\begin{align*}
    f(x,y) = \frac{xy^3}{x^2 + y^6}\ \ \ \ \ \ \text{ if } f(x,y) \neq (0,0)
\end{align*}
\textit{(a) Show that the directional derivatives of $f$ at $(0,0)$ exist in every direction.  What is its G\^{a}teaux derivatives at $(0, 0)$? \\
(b) Show that $f$ is not Fr\'{e}chet differentiable at $(0,0)$.  (HINT.  A Fr\'{e}chet differentiable function must be continuous.)} \\

Let $u = \langle u_1, u_2 \rangle$ and consider $D_u f(0, 0)$
\begin{align*}
    D_u f(0, 0) &= \lim_{\varepsilon \rightarrow 0}\frac{f(0 + \varepsilon u_1, 0 + \varepsilon u_2) - f(0, 0)}{\varepsilon} \\
    &= \lim_{\varepsilon \rightarrow 0} \qty[\frac{1}{\varepsilon} \frac{(\varepsilon u_1)(\varepsilon u_2)^3}{(\varepsilon u_1)^2 + (\varepsilon u_2)^6}] \\
    &= \lim_{\varepsilon \rightarrow 0} \qty[\varepsilon \frac{u_1 u_2^3}{u_1^2 + \varepsilon^4 u_2^6}] \\
    &= 0
\end{align*}
provided $u_1 \neq 0$.  If $u_1 = 0$,
\begin{align*}
    D_u f(0, 0) &= \lim_{\varepsilon \rightarrow 0}\frac{f(0, 0 + \varepsilon u_2) - f(0, 0)}{\varepsilon} \\
    &= \lim_{\varepsilon \rightarrow 0} \qty[\frac{1}{\varepsilon} \frac{(0)(\varepsilon u_2)^3}{0 + (\varepsilon u_2)^6}] \\
    &= \lim_{\varepsilon \rightarrow 0}\qty[0] \\
    &= 0
\end{align*}
Thus all directional derivatives exist and the G\^{a}teaux derivative $\grad f = f_x\textbf{i} + f_y\textbf{j} = 0\textbf{i} + 0\textbf{j} = \textbf{0}$.

If $f$ is Fr\'{e}chet differentiable, $f$ must be continuous.  However, $f$ is not continuous because consider approaching $(0,0)$ along the curve $y = x$.  Then
\begin{align*}
    \lim_{x\rightarrow 0} f(x,y) &= \lim_{x\rightarrow 0} f(x,x) = \lim_{x \rightarrow 0} \frac{x^4}{x^2 + x^6} = \lim_{x\rightarrow 0} \frac{x^2}{1 + x^4} = 0
\end{align*}
Now consider approaching $(0, 0)$ along the curve $y = \sqrt[3]{x}$.  Then
\begin{align*}
    \lim_{x\rightarrow 0} f(x,y) &= \lim_{x\rightarrow 0} f(x,\sqrt[3]{x}) = \lim_{x \rightarrow 0} \frac{x^2}{x^2 + x^2} = \lim_{x \rightarrow 0} \frac{1}{2} = \frac{1}{2}
\end{align*}
Since $0 \neq \frac{1}{2}$, $f$ is not continuous, and thus $f$ is not Fr\'{e}chet differentiable at $(0, 0)$.

\section*{Problem 2}
\textit{Define $f,g\ :\ \mathbb{R}^2 \rightarrow \mathbb{R}$ by}
\begin{align*}
    f(x,y) = x^2 + y^2,\ \ \ \ \ \ g(x,y) = (y-1)^3 - x^2
\end{align*}
\textit{Find the minimum value of $f(x,y)$ subject to the constraint $g(x,y) = 0$.  Show that there does not exist any constant $\lambda$ such that $\grad f = \lambda \grad g$ at some point $(x,y) \in \mathbb{R}^2$.  Why does the method of Lagrange multipliers fail in this example?} \\

Note that $\grad g = \langle -2x, 3(y - 1)^2 \rangle$ and $\grad f = \langle 2x, 2y \rangle$.  Also note that the method of Lagrange multipliers cannot be used if $\grad g = \textbf{0}$.

Now assume that $\grad f = \lambda \grad g$.  Then
\begin{align*}
    f_x &= \lambda g_x\ \ \ \ \text{and} \\
    f_y &= \lambda g_y
\end{align*}
Thus,
\begin{align*}
    2x &= \lambda(-2x) \ \ \ \ \text{and} \\
    2y &= \lambda(3(y - 1)^2)
\end{align*}
The first equation gives either $\lambda = -1$ or $x = 0$.  If $x = 0$, then $g = 0 \implies y = 1$.  However, $(x,y) = (0,1) \implies \grad g = \textbf{0}$.  Thus $\lambda = -1$.  The second equation gives
\begin{align*}
    2y &= -3(y^2 - 2y + 1) \\
    \implies 3y^2 - 4y + 3 &= 0 \\
    \implies y &= \frac{4 \pm \sqrt{-20}}{6} \not\in \mathbb{R}
\end{align*}
which is not possible since $f,g\ :\ \mathbb{R}^2 \rightarrow \mathbb{R}$.  Thus the Lagrange multiplier method does not work in this case.

% To find the minimum, note that $g(x,y) = 0 \implies x^2 = (y - 1)^3$.  Thus,
% \begin{align*}
%     f(x,y) = \hat{f}(y) &= f(\sqrt{(y-1)^3}, y) \\
%     &= (y-1)^3 + y^2 \\
%     &= y^3 - y^2 + 3y - 1 \\
%     \implies \hat{f}'(y) &= 3y^2 - 2y + 3
% \end{align*}
% $\hat{f}'(y) = 0$ when $y = \frac{2 \pm \sqrt{4 - 36}}{6}$

\section*{Problem 3}
\textit{Derive the Euler-Lagrange equation for a functional of the form}
\begin{align*}
    J(u) = \int_a^b F(x, u, u', u'') \dd x
\end{align*}
\textit{What are the natural boundary conditions for this functional?} \\

Assume $u$ minimizes the functional.  Then
\begin{align*}
    \dd J(u, \phi) = \frac{\dd}{\dd \varepsilon} \int_a^b F(x, u + \varepsilon \phi, u' + \varepsilon \phi', u'' + \varepsilon \phi'')\Big|_{\varepsilon = 0} &= 0 \\
    \implies \int_a^b \qty[F_u(x, u, u', u'')\phi + F_{u'}(x, u, u', u'')\phi' + F_{u''}(x, u, u', u'')\phi''] \dd x &= 0
\end{align*}
By integrating by parts, we get
\begin{align*}
    \int_a^b F_{u'}(x, u, u', u'')\phi' \dd x &= \qty[F_{u'}(x, u, u', u'')\phi]_a^b - \int_a^b \qty(\frac{\dd}{\dd x}F_{u'}(x, u, u', u''))\phi \dd x
\end{align*}
and
\begin{align*}
    \int_a^b F_{u''}(x, u, u', u'')\phi'' \dd x &= \qty[F_{u''}(x, u, u', u'')\phi']_a^b - \int_a^b \qty(\frac{\dd}{\dd x}F_{u''}(x, u, u', u''))\phi' \dd x
\end{align*}
Again, by integrating by parts,
\begin{align*}
    \int_a^b \qty(\frac{\dd}{\dd x}F_{u''}(x, u, u', u''))\phi' \dd x &= \qty[\qty(\frac{\dd}{\dd x}F_{u''}(x, u, u', u''))\phi]_a^b - \int_a^b \qty(\frac{\dd^2}{\dd x^2}F_{u''}(x, u, u', u''))\phi \dd x
\end{align*}
Thus,
\begin{align*}
    \int_a^b \qty[F_u \phi - \qty(\frac{\dd}{\dd x}F_{u'})\phi + \qty(\frac{\dd}{\dd x^2}F_{u''})\phi]\dd x + \qty[F_{u'}\phi + F_{u''}\phi' - \qty(\frac{\dd}{\dd x}F_{u''})\phi]_a^b = 0
\end{align*}
Thus, given natural boundary conditions, the Euler-Lagrange equation is
\begin{align*}
    F_u - \frac{\dd}{\dd x}F_{u'} + \frac{\dd^2}{\dd x^2}F_{u''} = 0
\end{align*}
The required boundary conditions are
\begin{align*}
    \qty[\qty(F_{u'} - \frac{\dd}{\dd x}F_{u''})\phi]_a^b + \qty[F_{u''}\phi'] = 0
\end{align*}
The most ``natural'' boundary conditions are
\begin{align*}
    \qty[F_{u'} - \frac{\dd}{\dd x}F_{u''}]\Bigg|_{x = a} = \qty[F_{u'} - \frac{\dd}{\dd x}F_{u''}]\Bigg|_{x = b} = \qty(F_{u''})\Big|_{x = a} = \qty(F_{u''})\Big|_{x = b} = 0
\end{align*}

\section*{Problem 4}
\textit{A curve $y = u(x)$ with $a \leq x \leq b$, $u(x) > 0$, and $u(a) = u_0$, $u(b) = u_1$ is rotated about the $x$-axis.  Find the curve that minimizes the area of the surface of revolution,}
\begin{align*}
    J(u) = \int_a^b u \sqrt{1 + (u')^2}\dd x
\end{align*} \\

Let $F(x, u, u') = F(u, u') = u\sqrt{1 + (u')^2}$.  Since $J(u) = \int_a^b F(u, u') \dd x$, and since $F(u, u')$ is explicitly independent of $x$, we can guarantee
\begin{align*}
    F - u'F_{u'} = C \in \mathbb{R}
\end{align*}
Note
\begin{align*}
    F_{u'} = \frac{u u'}{\sqrt{1 + (u')^2}}
\end{align*}
Thus,
\begin{align*}
    u\sqrt{1 + (u')^2} - u'\frac{u u'}{\sqrt{1 + (u')^2}} &= C \\
    \implies \frac{u(1 + (u')^2)}{\sqrt{1 + (u')^2}} - \frac{u (u')^2}{\sqrt{1 + (u')^2}} &= C \\
    \implies u &= C\sqrt{1 + (u')^2} \\
    \implies \frac{\dd u}{\dd x} &= \sqrt{\qty(\frac{u}{C})^2 - 1}
\end{align*}
This is a separable differential equation, so
\begin{align*}
    C\int \frac{1}{\sqrt{u^2 - C^2}}\dd u &= \int \dd x \\
    \implies C\ln\qty(\sqrt{u^2 - C^2} + u) &= x + K
\end{align*}
where $K \in \mathbb{R}$.  We obtain a system of equations by noting the Dirichlet boundary conditions $u(a) = u_0$, $u(b) = u_1$:
\begin{align*}
    C\ln\qty(\sqrt{u_0^2 - C^2} + u_0) &= a + K \\
    C\ln\qty(\sqrt{u_1^2 - C^2} + u_1) &= b + K
\end{align*}
We can solve this system for $C = C(u_0, u_1, a, b)$ and $K = K(u_0, u_1, a, b)$ in terms of $u_0$, $u_1$, $a$, and $b$ to obtain the implicit curve
\begin{align*}
    C\ln\qty(\sqrt{u^2 - C^2} + u) &= x + K
\end{align*}

\section*{Problem 5}
\textit{Let $X$ be the space of smooth functions $u\ :\ [0,1] \rightarrow \mathbb{R}$ such that $u(0) = 0$, $u(1) = 0$.  Define functionals $J,K\ :\ X \rightarrow \mathbb{R}$ by}
\begin{align*}
    J(u) = \frac{1}{2}\int_0^1(u')^2 \dd x, \ \ \ \ \ \ K(u) = \frac{1}{2}\int_0^1 u^2 \dd x
\end{align*}
\textit{(a) Introduce a Lagrange multiplier and write down the Euler-Lagrange equation for extremals in $X$ of the functional $J(u)$ subject to the constraint $K(u) = 1$.\\
(b) Solve the eigenvalue problem in (a) and find all of the extremals.  Which one minimizes $J(u)$?} \\

Define $L\ :\ X \rightarrow \mathbb{R}$ by
\begin{align*}
    L(u) = K(u) - 1 = \frac{1}{2}\int_0^1 u^2 \dd x - \int_0^1 \dd x = \int_0^1 \qty(\frac{1}{2}u^2 - 1) \dd x
\end{align*}
Then the constraint $K(u) = 1$ is equivalent to $L(u) = 0$.  Thus we consider
\begin{align*}
    \frac{\delta J}{\delta u} &= \lambda \frac{\delta L}{\delta u}
\end{align*}
Define $F(x, u, u') = F(u') = \frac{1}{2}(u')^2$ and $G(x, u, u') = G(u) = \frac{1}{2}u^2 - 1$.
Then
\begin{align*}
    \qty[-\frac{\dd}{\dd x}F_{u'} + F_u] &= \lambda\qty[-\frac{\dd}{\dd x}G_{u'} + G_u]
\end{align*}
Note that $F_{u'} = u'$, $F_u = 0$, $G_{u'} = 0$ and $G_u = u$.  Thus,
\begin{align*}
    -\frac{\dd}{\dd x} u' &= \lambda u \\
    \implies u'' + \lambda u &= 0 \\
    \implies u(x) &= a \exp(\sqrt{-\lambda}x) + b \exp(-\sqrt{-\lambda}x)
\end{align*}
If $\lambda < 0$, then $u(0) = 0$ implies $0 = a + b$ and $u(1) = 0$ implies
\begin{align*}
    0 = a\sinh(\sqrt{\lambda})
\end{align*}
Thus either $a = 0$ or $\lambda = 0$.  But $\lambda < 0 \implies \lambda \neq 0$.  Thus $a = 0$, and thus $b = 0$.  So
\begin{align*}
    u \equiv \textbf{0}
\end{align*}
But $L(\textbf{0}) = \int_0^1 -1 \dd x = -1 \neq 0$, which contradicts our constrain requirement.  Thus $\lambda > 0$.  This gives
\begin{align*}
    u(x) = a\sin(\sqrt{\lambda}x) + b\cos(\sqrt{\lambda}x)
\end{align*}
Using the boundary requirements,
\begin{align*}
    0 &= b\ \ \ \ \ \ \text{and} \\
    0 &= a\sin(\sqrt{\lambda})
\end{align*}
Thus $\lambda = \pi^2$ or $a = 0$.  However, if $a = 0$ then $u \equiv \textbf{0}$, which is a contradiction as showed above.  Thus $\lambda = \pi^2$.  So,
\begin{align*}
    u(x) = a\sin(\pi x)
\end{align*}
Since $L(u) = 0$,
\begin{align*}
    \frac{1}{2}\int_0^1\qty(a\sin(\pi x))^2 \dd x - 1 &= 0 \\
    \implies \frac{a^2}{2}\int_0^1 (\sin(\pi x))^2 \dd x - 1 &= 0 \\
    \implies \frac{a^2}{2} \frac{1}{2} - 1 &= 0 \\
    \implies a^2 = 4 \\ 
    \implies a = \pm 2
\end{align*}
Thus $u_+(x) = 2\sin(\pi x)$ and $u_-(x) = -2\sin(\pi x)$ minimize $J(u)$ subject to $L(u) = 0$.  Also note that $(u_+')^2 = (u_-')^2 = 4\pi^2\qty(\cos(\pi x))^2$ and so $J(u_+) = J(u_-) = \pi^2$.

\section*{Problem 6}
\textit{(a) Make a change of variable $x = \phi(t)$, $v(t) = u(\phi(t))$, where $\phi'(t) > 0$, in the functional}
\begin{align*}
    J(u) = \int_a^b F(x, u, u')\dd x
\end{align*}
\textit{Show that $J(u) = K(v)$ where $K(v)$ has the form}
\begin{align*}
    K(v) = \int_c^d G(t, v, v')\dd t
\end{align*}
\textit{and express $G$ in terms of $F$ and $\phi$.\\
(b) Show that the Euler-Lagrange equation for $K(v)$ is the same as what you get by changing variables in the Euler-Lagrange equation for $J(u)$.} \\

Let $x = \phi(t)$ and $v(t) = u(\phi(t))$ where $\phi'(t) > 0$.  Then note, by the chain rule, $\dd x = \phi'(t) \dd t$ and $v'(t) = u_\phi \phi'(t)$.  Thus, $x = a \implies t = c = \phi^{-1}(a)$ and $x = b \implies t = d = \phi^{-1}(b)$.  Also,
\begin{align*}
    F(x, u, u') = F\qty(\phi(t), v(t), \frac{v'(t)}{\phi'(t)})
\end{align*}
So let $G(t, v, v') = \phi'(t)F\qty(\phi(t), v(t) \frac{v'(t)}{\phi'(t)})$.  Then
\begin{align*}
    J(u) = \int_a^b F(x, u, u') = \int_c^d G(t, v, v')\dd t = K(v)
\end{align*}
Also,
\begin{align*}
    -\frac{\dd}{\dd x}F_{u'} + F_u &= 0 \\
    \implies -\frac{1}{\phi'}\frac{\dd}{\dd t}F_{\frac{v'}{\phi'}}\qty(\phi, v, \frac{v'}{\phi'}) + F_v\qty(\phi, v, \frac{v'}{\phi'}) &= 0 \\
    \implies -\frac{1}{\phi'}\frac{\dd}{\dd t}G_{v'} + \frac{1}{\phi'}G_v &= 0 \\
    \implies -\frac{\dd}{\dd t}G_{v'} + G_v &= 0
\end{align*}
since $\phi' > 0$.

\end{document}
