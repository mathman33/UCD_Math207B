\documentclass{article} % A4 paper and 11pt font size
\setcounter{secnumdepth}{0}

\usepackage{amssymb, amsmath, amsfonts}
\usepackage{moreverb}
\usepackage{graphicx}
\usepackage{enumerate}
\usepackage{graphics}
\usepackage[margin=1.25in]{geometry}
\usepackage{color}
\usepackage{tocloft}
\renewcommand{\cftsecleader}{\cftdotfill{\cftdotsep}}
\usepackage{array}
\usepackage{float}
\usepackage{hyperref}
\usepackage{textcomp}
\usepackage[makeroom]{cancel}
\usepackage{bbold}
\usepackage{alltt}
\usepackage{physics}
\usepackage{mathtools}
\usepackage{amsthm}
\usepackage{tikz}
\usetikzlibrary{positioning}
\usetikzlibrary{arrows}
\usepackage{pgfplots}
\usepackage{bigints}
\allowdisplaybreaks
\pgfplotsset{compat=1.12}

\theoremstyle{plain}
\newtheorem*{theorem*}{Theorem}
\newtheorem{theorem}{Theorem}
\newtheorem*{lemma*}{Lemma}
\newtheorem{lemma}{Lemma}

\newenvironment{definition}[1][Definition]{\begin{trivlist}
\item[\hskip \labelsep {\bfseries #1}]}{\end{trivlist}}

\newcommand{\E}{\varepsilon}
\def\Rl{\mathbb{R}}
\def\Cx{\mathbb{C}}

\usepackage[T1]{fontenc} % Use 8-bit encoding that has 256 glyphs
\usepackage{fourier} % Use the Adobe Utopia font for the document - comment this line to return to the LaTeX default
\usepackage[english]{babel} % English language/hyphenation

\usepackage{sectsty} % Allows customizing section commands
\allsectionsfont{\centering \normalfont\scshape} % Make all sections centered, the default font and small caps

\usepackage{fancyhdr} % Custom headers and footers
\pagestyle{fancy} % Makes all pages in the document conform to the custom headers and footers
\fancyhead[L]{\bf Sam Fleischer}
\fancyhead[C]{\bf UC Davis \\ Applied Mathematics (MAT207B)} % No page header - if you want one, create it in the same way as the footers below
\fancyhead[R]{\bf Winter 2016}

\fancyfoot[L]{\bf } % Empty left footer
\fancyfoot[C]{\bf \thepage} % Empty center footer
\fancyfoot[R]{\bf } % Page numbering for right footer
\renewcommand{\headrulewidth}{0pt} % Remove header underlines
\renewcommand{\footrulewidth}{0pt} % Remove footer underlines
\setlength{\headheight}{25pt} % Customize the height of the header

\newcommand{\problem}[1]{
\vspace{.375cm}
\begin{minipage}{\textwidth}
    \begin{center}
        \noindent\rule{5cm}{1pt}
    \end{center}
    \section{\bf #1}
    \begin{center}
        \noindent\rule{5cm}{1pt}
    \end{center}
    \vspace{0.25cm}
\end{minipage}
}

\newcommand{\VEC}[2]{\left\langle #1, #2 \right\rangle}

% \numberwithin{equation}{section} % Number equations within sections (i.e. 1.1, 1.2, 2.1, 2.2 instead of 1, 2, 3, 4)
% \numberwithin{figure}{section} % Number figures within sections (i.e. 1.1, 1.2, 2.1, 2.2 instead of 1, 2, 3, 4)
% \numberwithin{table}{section} % Number tables within sections (i.e. 1.1, 1.2, 2.1, 2.2 instead of 1, 2, 3, 4)

\setlength\parindent{0pt} % Removes all indentation from paragraphs - comment this line for an assignment with lots of text

\newcommand{\horrule}[1]{\rule{\linewidth}{#1}} % Create horizontal rule command with 1 argument of height

\title{ 
\normalfont \normalsize 
\textsc{UC Davis, Applied Mathematics (MAT207B), Winter 2016} \\ [25pt] % Your university, school and/or department name(s)
\horrule{2pt} \\[0.4cm] % Thin top horizontal rule
\Huge Homework \#4 \\ % The assignment title
\horrule{2pt} \\[0.5cm] % Thick bottom horizontal rule
}

\author{\huge Sam Fleischer} % Your name

\date{February 12, 2016} % Today's date or a custom date

\begin{document}\thispagestyle{empty}

\maketitle % Print the title

\makeatletter
\@starttoc{toc}
\makeatother

\pagebreak

\problem{Problem 1}
\emph{The following nonhomogeneous IBVP describes heat flow in a rod whose ends are held at temperatures $u_0$, $u_1$:}
\begin{equation}
    \label{Dirichlet_lin}
    \begin{aligned}
        &u_t = u_{xx}\ \ \ \ \ \ \ \ \ 0 < x < 1,\ \ \ \ \ t > 0 \\
        &u(0, t) = u_0, \ \ \ \ \ \ \ u(1,t) = u_1 \\
        &u(x,0) = f(x)
    \end{aligned}
\end{equation}
\begin{enumerate}[\bf (a)]  
    \item
        \emph{Find the steady state temperature $U(x)$ that satisfies}
        \begin{align*}
            &U_{xx} = 0 \ \ \ \ \ 0 < x < 1 \\
            &U(0) = u_0,\ \ \ \ \ \ U(1) = u_1
        \end{align*}

        $U_{xx} = 0$ implies $U(x)$ is a linear function.
        \begin{align*}
            U(x) = a + bx
        \end{align*}
        The boundary conditions imply
        \begin{align*}
            U(x) = u_0 + (u_1 - u_0)x
        \end{align*}
    \item
        \emph{Write $u(x,t) = U(x) + v(x,t)$ and find the corresponding IBVP for $v$.  Use separation of variables to solve for $v$ and hence $u$.} \\

        If $u(x, t) = U(x) + v(x, t)$, then $u(0, t) = u_0 = U(0) + v(0, t)$ implies $v(0, t) = 0$.  Similarly, $v(1,t) = 0$.  Also, $u(x,0) = f(x) = U(x) + v(x,0)$ implies $v(x,0) = f(x) - U(x)$.  Lastly, partial differential equation $u_t = u_{xx}$ implies $v_t = U''(x) + v_{xx}$ but since $U'' = 0$, then $v_t = v_{xx}$.  The problem becomes
        \begin{equation}
            \label{Dirichlet_0}
            \begin{aligned}
                &v_t = v_{xx}\ \ \ \ \ \ \ \ \ 0 < x < 1,\ \ \ \ \ t > 0 \\
                &v(0, t) = 0, \ \ \ \ \ \ \ v(1,t) = 0 \\
                &v(x,0) = f(x) - U(x)
            \end{aligned}
        \end{equation}

        By our last homework, the solution of (\ref{Dirichlet_0}) is
        \begin{align*}
            v(x,t) = \sum_{n=1}^\infty c_n e^{-n^2\pi^2 t}\sin\qty(n\pi x)
        \end{align*}
        where
        \begin{align*}
            c_n = 2\int_0^1\qty(f(x) - U(x))\sin\qty(n\pi x) \dd x
        \end{align*}
        Thus the solution to (\ref{Dirichlet_lin}) is
        \begin{align*}
            u(x, t) = u_0 + (u_1 - u_0)x + \sum_{n=1}^\infty c_n e^{-n^2\pi^2 t}\sin\qty(n\pi x)
        \end{align*}
    \item
        \emph{How does $u(x,t)$ behave as $t \rightarrow \infty$?} \\

        As $t \rightarrow \infty$, the coefficients of the $\sin$ series decay hyper-exponentially, and thus the solution rapidly approaches the linear function $U(x)$, i.e.
        \begin{align*}
            \lim_{t\rightarrow \infty} u(x,t) = U(x)
        \end{align*}
\end{enumerate}

\problem{Problem 2}
\emph{Define a first-order differential operator with complex coefficients acting $L^2(0, 2\pi)$ by $$A = -i \frac{\dd}{\dd x}.$$}
\begin{enumerate}[\bf (a)]
    \item
        \emph{Show that $A$ is formally self-adjoint.}
        \begin{align*}
            \VEC{u}{Av} &= \int_0^{2\pi} -i\overline{u}v'\dd x \\
            &= -i\qty[\qty(\overline{u}v)_0^{2\pi} - \int_0^{2\pi}\overline{u'}v\dd x] \\
            &= -i\qty(\overline{u(2\pi)}v(2\pi) - \overline{u(0)}v(0)) + \int_0^{2\pi} i \overline{u'}v \dd x \\
            &= -i\qty(\overline{u(2\pi)}v(2\pi) - \overline{u(0)}v(0)) + \int_0^{2\pi} \overline{-iu'}v \dd x \\
            &= -i\qty(\overline{u(2\pi)}v(2\pi) - \overline{u(0)}v(0)) + \VEC{Au}{v}
        \end{align*}
        Thus, given adequate boundary conditions, $A$ is self-adjoint, i.e. $A$ is formally self-adjoint.
    \item
        \emph{Show that $A$ with periodic boundary conditions $u(0) = u(2\pi)$ is self-adjoint, and find the eigenvalues and eigenfunctions of the corresponding eigenvalue problem $$ -iu' = \lambda u, \qquad u(0) = u(2\pi).$$}

        If $u(0) = u(2\pi)$, then $\overline{0} = \overline{u(2\pi)}$.  Thus,
        \begin{align*}
            \VEC{u}{Av} = -i\qty(u(0)\qty[v(2\pi) - v(0)]) + \VEC{Au}{v}
        \end{align*}
        and so the adjoint boundary condition is $v(0) = v(2\pi)$.  In this case, $A$ is self-adjoint.

        The solution to $-i u' = \lambda u$ is
        \begin{align*}
            u(x) = c\exp(\lambda i x)
        \end{align*}
        for some constant $c$.  The periodicity of $u$ implies
        \begin{align*}
            c &= c\exp(\lambda i 2\pi) \implies \lambda = n,\ \ \ \ \ n \in \mathbb{Z}
        \end{align*}
    \item
        \emph{What are the adjoint boundary conditions to the Dirichlet condition $u(0) = 0$ at $x = 0$?  Is $A$ with this Dirichlet boundary condition self-adjoint?  Find all eigenvalues and eigenfunctions of the corresponding eigenvalue problem $$ -iu' = \lambda u, \qquad u(0) = 0.$$  How does your result compare with the properties of finite-dimensional eigenvalue problems for matrices?} \\

        If $u(0) = 0$ then $\overline{u(0)} = 0$ and 
        \begin{align*}
            \VEC{u}{Av} = -i\overline{u(2\pi)}v(2\pi) + \VEC{Au}{v}
        \end{align*}
        Thus the adjoint condition is $v(2\pi) = 0$, and in this case, $A$ is self-adjoint.  The solution to $-iu' = \lambda u$ is
        \begin{align*}
            u(x) = c\exp(\lambda i x)
        \end{align*}
        for some constant $c$.  The boundary condition $u(0) = 0$ implies $c = 0$, and thus $u \equiv 0$.  Since eigenvalues cannot have $0$ eigenfunctions, there are no eigenvalues for this eigenvalue problem.  This does not happen in finite-dimensional eigenvalue problems since all matrices (over algebraically closed fields, like $\Cx$) have at least one eigenvalue.
\end{enumerate}

\problem{Problem 3}
\emph{Let $A$ be a Sturm-Liouville operator, given by $$Au = -(pu')' + qu,$$ acting in $L^2(a,b)$.  Verify that $A$ with the Robin boundary conditions $$\alpha u'(a) + u(a) = 0, \qquad u'(b) + \beta u(b) = 0$$ is self adjoint.}
\begin{align*}
    \VEC{u}{Av} &= \int_a^b \qty[u\qty(-(pv')' + qv)]\dd x \\
    &= -\int_a^b u p' v' \dd x - \int_a^b u p v'' \dd x + \int_a^b u q v \dd x
\end{align*}
We can integrate the middle integral by parts:
\begin{align*}
    \int_a^b u p v'' \dd x &= \qty[u p v']_a^b - \int_a^b (u p' v' + u' p v') \dd x \\
    \implies \VEC{u}{Av} &= -\qty[upv']_a^b + \int_a^b u' p v'\dd x + \int_a^b u q v \dd x \\
\end{align*}
Again, we can integrate the leftmost integral by parts:
\begin{align*}
    \int_a^b u' p v'\dd x &= \qty[u' p v]_a^b - \int_a^b (u'' p v - u' p' v) \dd x \\
    \implies \VEC{u}{Av} &= \qty[p(u'v - uv')]_a^b + \int_a^b (-(u'' p v + u' p' v) + u q v) \dd x \\
    &= \qty[p(u'v - uv')]_a^b + \int_a^b \qty[\qty(-(pu')' + qu)v]\dd x \\
    &= \qty[p(u'v - uv')]_a^b + \VEC{Au}{v}
\end{align*}
The Robin boundary conditions imply $u'(a) = -\frac{u(a)}{\alpha}$ and $u'(b) = -\beta u(b)$.  If we impose the adjoint condition $$\alpha v'(a) + v(a) = 0, \qquad v'(b) + \beta v(b) = 0,$$ then
\begin{align*}
    \qty[p(u'v - uv')]_a^b &= p(b)\qty(u'(b)v(b) - u(b)v'(b)) - p(a)\qty(u'(a)v(a) - u(a)v'(a)) \\
    &= p(b)\qty(-\beta u(b)v(b) + \beta u(b)v(b)) - p(a)\qty(-\frac{1}{\alpha}u(a)v(a) + \frac{1}{\alpha}u(a)v(a)) \\
    &= 0
\end{align*}
and thus $A$ is self-adjoint.

\problem{Problem 4}
\emph{Show that the eigenvalues of the Sturm-Liouville problem}
\begin{align*}
    &-u'' = \lambda u \qquad 0 < x < 1 \\ &u(0) = 0, \qquad u'(1) + \beta u(1) = 0
\end{align*}
\emph{are given by $\lambda = k^2$ where $k > 0$ satisfies the equation $$ \beta \tan k + k = 0.$$  Show graphically that there is an infinite sequence of simple eigenvalues $\lambda_1 < \lambda_2 < \dots < \lambda_n < \dots$ with $\lambda_n \rightarrow \infty$ as $n \rightarrow \infty$.  What is the asymptotic behavior of $\lambda_n$ as $n \rightarrow \infty$?} \\

If $\lambda = 0$, the solutions to $-u'' = \lambda u$ are
\begin{align*}
    u(x) = c_1 + c_2 x
\end{align*}
Then $u(0) = 0 \implies c_1 = 0$, i.e.
\begin{align*}
    u(x) &= c_2 x \\
    u'(x) &= c_2
\end{align*}
The Robin boundary condition $u'(1) + \beta u(1) = 0$ implies
\begin{align*}
    0 &= c_2 + \beta c_2 = c_2(1 + \beta)
\end{align*}
If $\beta \neq -1$, then $c_2 = 0$ and thus $u \equiv 0$.  If $\lambda \neq 0$, the solutions to $-u'' = \lambda u$ are
\begin{align*}
    u(x) &= c_1\exp(\sqrt{\lambda}x) + c_2\exp(-\sqrt{\lambda}x)
\end{align*}
If $\lambda > 0$, then $u(0) = 0$ implies $c_1 = -c_2$, i.e.
\begin{align*}
    u(x) &= c_1\qty[\exp(\sqrt{\lambda}x) - \exp(-\sqrt{\lambda}x)] \\
    u'(x) &= c_1\sqrt{\lambda}\qty[\exp(\sqrt{\lambda}x) + \exp(-\sqrt{\lambda}x)]
\end{align*}
The Robin boundary condition $u'(1) + \beta u(1) = 0$ implies
\begin{align*}
    0 &= c_1\sqrt{\lambda}\qty[\exp(\sqrt{\lambda}) + \exp(-\sqrt{\lambda})] + \beta c_1 \qty[\exp(\sqrt{\lambda}) - \exp(-\sqrt{\lambda})] \\
    &= \qty(\sqrt{\lambda} + \beta)\exp(\sqrt{\lambda}) + \qty(\sqrt{\lambda} - \beta)\exp(-\sqrt{\lambda})
\end{align*}
If $\lambda < 0$, then the solution is given by
\begin{align*}
    u(x) = c_1\cos\qty(\sqrt{-\lambda}x) + c_2\sin\qty(\sqrt{-\lambda}x)
\end{align*}
The boundary condition $u(0) = 0$ implies $c_1 = 0$, i.e.
\begin{align*}
    u(x) &= c_2\sin\qty(\sqrt{-\lambda}x) \\
    u'(x) &= c_2\sqrt{-\lambda}\cos\qty(\sqrt{-\lambda}x)
\end{align*}
The Robin boundary condition $u'(1) + \beta u(1) = 0$ implies
\begin{align*}
    0 &= c_2\sqrt{-\lambda}\cos\qty(\sqrt{-\lambda}) + \beta c_2\sin\qty(\sqrt{-\lambda}) \\
    &= \sqrt{-\lambda} +\beta\tan\qty(\sqrt{-\lambda})\qquad \text{provided $c_2 \neq 0$} \\
    &= \beta\tan k + k
\end{align*}
where $k^2 = -\lambda$.  Let the solutions to $0 = \beta \tan k + k$ be $k_n$ where $\frac{\pi}{2} < |k_{n+1} - k_n| < \pi$ for all $n$.  Also, $k_n \rightarrow \infty$, and thus $\lambda_n \rightarrow \infty$ as $n \rightarrow \infty$.

\problem{Problem 5}
\emph{THe following IBVP describes heat flow in a rod whose left end is held at temperature $0$ and whose right end loses heat to the surroundings according to Newton's law of cooling (heat flux is proportional to the temperature difference):}
\begin{align*}
    &u_t = u_{xx} \qquad 0 < x < 1,\ \ \ t > 0 \\
    &u(0, t) = 0, \qquad u_x(1, t) = -\beta u(1,t) \\
    &u(x,0) = f(x)
\end{align*}
\emph{Solve this IBVP by the method of separation of variables.} \\

Assume the solution $u(x,t) = F(x)G(t)$.  Then $FG' = F''G$, which implies $\frac{G'}{G} = \frac{F''}{F} = \lambda$, i.e.
\begin{align*}
    G' - \lambda G = 0 \qquad \text{and} \qquad F'' - \lambda F = 0
\end{align*}
Then
\begin{align*}
    G(t) = c\exp[\lambda t]
\end{align*}
and, if $\lambda \neq 0$, then
\begin{align*}
    F(x) = c_1\exp[\sqrt{\lambda}x] + c_2\exp[-\sqrt{\lambda}x]
\end{align*}
If $\lambda = 0$, then
\begin{align*}
    F(x) = c_1 + c_2x
\end{align*}
The left boundary condition implies $F(0) = 0$, and so if $\lambda = 0$ then $c_1 = 0$.  Then $F(x) = c_2 x$.  The right boundary condition implies $F'(1) = -\beta F(1)$.  Thus $c_2 = -\beta c_2 \iff c_2(1 + \beta) = 0 \iff \beta = -1$ for nontrivial solutions $F$.  If $\lambda > 0$, then the left boundary condition implies $c_1 = -c_2$, and thus
\begin{align*}
    F(x) &= c_1\qty(\exp[\sqrt{\lambda}x] - \exp[-\sqrt{\lambda}x]) \\
    F'(x) &= c_1\sqrt{\lambda}\qty(\exp[\sqrt{\lambda}x] + \exp[-\sqrt{\lambda}x])
\end{align*}
The right boundary condition implies
\begin{align*}
    -c_1\beta \qty(\exp[\sqrt{\lambda}] - \exp[-\sqrt{\lambda}]) &= c_1\sqrt{\lambda}\qty(\exp[\sqrt{\lambda}] + \exp[-\sqrt{\lambda}]) \\
    \qty(\sqrt{\lambda} + \beta)\exp[\sqrt{\lambda}] + \qty(\sqrt{\lambda} - \beta)\exp[-\sqrt{\lambda}] &= 0
\end{align*}
If $\lambda < 0$, then
\begin{align*}
    F(x) &= c_1\sin\qty(\sqrt{-\lambda}x) + c_2\cos\qty(\sqrt{-\lambda}x)
\end{align*}
By number 4, 
\begin{align*}
    F(x) = c_1\sin\qty(\sqrt{-\lambda}x)
\end{align*}
Thus the general solution can be written as
\begin{align*}
    u(x,t) = \sum_{n=1}^\infty c_n\exp[\lambda_n t]\sin\qty(\sqrt{-\lambda_n}x)
\end{align*}
where $\lambda_n < 0$ for all $n$ and $\lambda_n \rightarrow -\infty$ as $n \rightarrow \infty$.

\end{document}
