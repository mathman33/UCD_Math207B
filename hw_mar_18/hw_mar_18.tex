\documentclass{article} % A4 paper and 11pt font size
\setcounter{secnumdepth}{0}

\usepackage{amssymb, amsmath, amsfonts}
\usepackage{moreverb}
\usepackage{graphicx}
\usepackage{enumerate}
\usepackage{graphics}
\usepackage[margin=1.25in]{geometry}
\usepackage{color}
\usepackage{tocloft}
\renewcommand{\cftsecleader}{\cftdotfill{\cftdotsep}}
\usepackage{array}
\usepackage{float}
\usepackage{hyperref}
\usepackage{textcomp}
\usepackage[makeroom]{cancel}
\usepackage{bbold}
\usepackage{alltt}
\usepackage{physics}
\usepackage{mathtools}
\usepackage{amsthm}
\usepackage{tikz}
\usetikzlibrary{positioning}
\usetikzlibrary{arrows}
\usepackage{pgfplots}
\usepackage{bigints}
\allowdisplaybreaks
\pgfplotsset{compat=1.12}

\theoremstyle{plain}
\newtheorem*{theorem*}{Theorem}
\newtheorem{theorem}{Theorem}
\newtheorem*{lemma*}{Lemma}
\newtheorem{lemma}{Lemma}

\newenvironment{definition}[1][Definition]{\begin{trivlist}
\item[\hskip \labelsep {\bfseries #1}]}{\end{trivlist}}

\newcommand{\E}{\varepsilon}
\def\Rl{\mathbb{R}}
\def\Cx{\mathbb{C}}

\usepackage[T1]{fontenc} % Use 8-bit encoding that has 256 glyphs
\usepackage{fourier} % Use the Adobe Utopia font for the document - comment this line to return to the LaTeX default
\usepackage[english]{babel} % English language/hyphenation

\usepackage{sectsty} % Allows customizing section commands
\allsectionsfont{\centering \normalfont\scshape} % Make all sections centered, the default font and small caps

\usepackage{fancyhdr} % Custom headers and footers
\pagestyle{fancy} % Makes all pages in the document conform to the custom headers and footers
\fancyhead[L]{\bf Sam Fleischer}
\fancyhead[C]{\bf UC Davis \\ Applied Mathematics (MAT207B)} % No page header - if you want one, create it in the same way as the footers below
\fancyhead[R]{\bf Winter 2016}

\fancyfoot[L]{\bf } % Empty left footer
\fancyfoot[C]{\bf \thepage} % Empty center footer
\fancyfoot[R]{\bf } % Page numbering for right footer
\renewcommand{\headrulewidth}{0pt} % Remove header underlines
\renewcommand{\footrulewidth}{0pt} % Remove footer underlines
\setlength{\headheight}{25pt} % Customize the height of the header

\newcommand{\problem}[1]{
\vspace{.375cm}
\begin{minipage}{\textwidth}
    \begin{center}
        \noindent\rule{5cm}{1pt}
    \end{center}
    \section{\bf #1}
    \begin{center}
        \noindent\rule{5cm}{1pt}
    \end{center}
    \vspace{0.25cm}
\end{minipage}
}

\newcommand{\VEC}[2]{\left\langle #1, #2 \right\rangle}

% \numberwithin{equation}{section} % Number equations within sections (i.e. 1.1, 1.2, 2.1, 2.2 instead of 1, 2, 3, 4)
% \numberwithin{figure}{section} % Number figures within sections (i.e. 1.1, 1.2, 2.1, 2.2 instead of 1, 2, 3, 4)
% \numberwithin{table}{section} % Number tables within sections (i.e. 1.1, 1.2, 2.1, 2.2 instead of 1, 2, 3, 4)

\setlength\parindent{0pt} % Removes all indentation from paragraphs - comment this line for an assignment with lots of text

\newcommand{\horrule}[1]{\rule{\linewidth}{#1}} % Create horizontal rule command with 1 argument of height

\title{ 
\normalfont \normalsize 
\textsc{UC Davis, Applied Mathematics (MAT207B), Winter 2016} \\ [25pt] % Your university, school and/or department name(s)
\horrule{2pt} \\[0.4cm] % Thin top horizontal rule
\Huge Homework \#7 \\ % The assignment title
\horrule{2pt} \\[0.5cm] % Thick bottom horizontal rule
}

\author{\huge Sam Fleischer} % Your name

\date{March 18, 2016} % Today's date or a custom date

\begin{document}\thispagestyle{empty}

\maketitle % Print the title

\makeatletter
\@starttoc{toc}
\makeatother

\pagebreak

%%%%%%%%%%%%%%%%%%%%%%%%%%%%%
\problem{Problem 1}
\begin{enumerate}[\it (a)]
    \item
        \emph{Use separation of variables to find the eigenvalues $\lambda$ and eigenfunctions $u(x,y)$ of the Dirichlet Laplacian on the unit square that satisfy}
        \begin{align*}
            -\qty(u_{xx} + u_{yy}) = \lambda u \qquad 0 < x < 1,\ 0 < y < 1 \\
            u(x,0) = 0,\qquad u(x,1) = 0 \qquad 0 \leq x \leq 1 \\
            u(0,y) = 0,\qquad u(1,y) = 0 \qquad 0 \leq y \leq 1.
        \end{align*}

        Suppose $u(x,y) = F(x)G(y)$.  Then
        \begin{align*}
            -F''G - FG'' = \lambda FG \qquad \implies \qquad -F''G = F\qty(G'' + \lambda G) \\
            \implies -\frac{F''}{F} = \frac{G''}{G} + \lambda
        \end{align*}
        Since the left hand side is a function of $x$ and the right hand side is a function of $y$, then they can only be equal if
        \begin{align*}
            -\frac{F''}{F} = \frac{G''}{G} + \lambda = \mu
        \end{align*}
        where $\mu$ is a constant.  Note the boundary conditions imply
        \begin{align*}
            F(0) = F(1) = G(0) = G(1) = 0
        \end{align*}
        thus $F'' + \mu F = 0$ and the homogeneous Dirichlet boundary conditions imply $\mu > 0$ and
        \begin{align*}
            F(x) = A\sin(\sqrt{\mu}x)
        \end{align*}
        The condition $F(1) = 0$ implies
        \begin{align*}
            0 = A\sin(\sqrt{\mu}) \qquad \implies \qquad \mu = \pi^2 n^2
        \end{align*}
        for $n \geq 1$.  Then $G'' + (\lambda - \mu)G = 0$ and the homogeneous Dirichlet boundary conditions imply $\lambda - \mu > 0$ and
        \begin{align*}
            G(y) = B\sin(\sqrt{\lambda - \mu}y)
        \end{align*}
        The condition $G(1) = 0$ implies
        \begin{align*}
            0 = B\sin(\sqrt{\lambda - \mu}) \qquad \implies \qquad \lambda - \mu = \pi^2 m^2 \qquad \implies \qquad \lambda = \pi^2(n^2 + m^2)
        \end{align*}
        for $n, m \geq 1$.  Thus the solution to $-(u_{xx} + u_{yy}) = \lambda u$ is
        \begin{align*}
            \sum_{n,m \geq 1} A_{n,m}\sin(n\pi x)\sin(m\pi y)
        \end{align*}
        for some constants $A_{n,m}$.  Then the eigenvalues of
        \begin{align*}
            -\laplacian u = \lambda u; \qquad u(x,0) = u(x,1) = u(0,y) = u(1,y) 0
        \end{align*}
        are $\lambda = \pi^2(m^2 + n^2)$ for all $n,m \geq 1$.
    \item
        \emph{What is the smallest eigenvalue that is not a simple eigenvalue?} \\

        The smallest eigenvalue that is not a simple eigenvalue is $5\pi^2$ since this can be acheived when $n = 1$, $m = 2$ or when $n = 2$, $m = 1$.  The only smaller eigenvalue is $2\pi^2$, but that is simple since it can only be acheived when $n = m = 1$.
\end{enumerate}









%%%%%%%%%%%%%%%%%%%%%%%%%%%%%
\problem{Problem 2}
\begin{enumerate}[\it (a)]
    \item
        \emph{Let $\vec{x} = (x,y)$, $\vec{\xi} = (\xi, \eta)$, and $\vec{\xi^*} = (\xi, -\eta)$ where $\eta > 0$.  Show that $$G\qty(\vec{x}, \vec{\xi}) = - \frac{1}{2\pi}\log\qty(\frac{\left|\vec{x} - \vec{\xi}\right|}{\left|\vec{x} - \vec{\xi^*}\right|})$$ is the solution of}
        \begin{align*}
            -\qty(G_{xx} + G_{yy}) = \delta\qty(\vec{x} - \vec{\xi}) \qquad \text{ in } - \infty < x < \infty, \qquad y > 0 \\
            G\qty(\vec{x}, \vec{\xi}) = 0 \qquad \text{ on } y = 0.
        \end{align*}

        \begin{align*}
            -\laplacian G &= -\laplacian \qty[-\frac{1}{2\pi}\log\qty(\frac{\left|\vec{x} - \vec{\xi}\right|}{\left|\vec{x} - \vec{\xi^*}\right|})] \\
            &= -\laplacian \qty[- \frac{1}{2\pi}\log\left|\vec{x} - \vec{\xi}\right| - \qty(-\frac{1}{2\pi}\log\left|\vec{x} - \vec{\xi^*}\right|)] \\
            &= -\laplacian\qty[G_F\qty(\left|\vec{x} - \vec{\xi}\right|) - G_F\qty(\left|\vec{x} - \vec{\xi^*}\right|)]
        \end{align*}
        where
        \begin{align*}
            G_F(r) = -\frac{1}{2\pi}\log r
        \end{align*}
        is the free-space Green's function in two dimensions.  By definition,
        \begin{align*}
            -\laplacian G_F\qty(\left|\vec{x} - \vec{\xi}\right|) = \delta\qty(\vec{x} - \vec{\xi}) \qquad \text{and} \qquad -\laplacian G_F\qty(\left|\vec{x} - \vec{\xi^*}\right|) = \delta\qty(\vec{x} - \vec{\xi^*}).
        \end{align*}
        But $\delta\qty(\vec{x} - \vec{\xi^*}) = 0$ for $y > 0$ since $\xi^* = (\xi, -\eta)$ where $\eta > 0$.  Thus,
        \begin{align*}
            -\laplacian G &= -\laplacian G_F\qty(\left|\vec{x} - \vec{\xi}\right|) + \cancelto{0}{\laplacian G_F\qty(\left|\vec{x} - \vec{\xi^*}\right|)} = \delta\qty(\vec{x} - \vec{\xi}) \qquad \text{for } (x,y) \in \{(x,y)\ :\ y > 0\}
        \end{align*}
        When $y = 0$, $\left|\vec{x} - \vec{\xi}\right| = \left|\vec{x} - \vec{\xi^*}\right|$, thus
        \begin{align*}
            \frac{\left|\vec{x} - \vec{\xi}\right|}{\left|\vec{x} - \vec{\xi^*}\right|} = 1 \qquad \implies \qquad G\qty(\vec{x}, \vec{\xi}) = -\frac{1}{2\pi}\log(1) = 0
        \end{align*}
    \item
        \emph{Write down the Green's function representation for the solution $u(x,y)$ of the Dirichlet problem for the Laplacian in the upper half plane}
        \begin{align*}
            u_{xx} + u_{yy} = 0 \qquad \text{ in } -\infty < x < \infty, \qquad y > 0 \\
            u(x,0) = f(x).
        \end{align*}
        \emph{You can assume that $u(x,y) \rightarrow 0$ sufficiently rapidly as $|(x,y)| \rightarrow \infty$.} \\

        We utilize Green's identity
        \begin{align*}
            \int_\Omega \qty(u\laplacian v - v\laplacian u )\dd \vec{\xi} = \int_{\partial \Omega} \qty(u\frac{\partial v}{\partial n\qty(\vec{\xi})} - v\frac{\partial u}{\partial n\qty(\vec{\xi})})\dd s\qty(\vec{\xi})
        \end{align*}
        by setting $\Omega = \{(x,y)\ :\ y > 0\}$ and $v = G$ where $G$ is the free-space Green's function which solves $-\laplacian G = \delta(x)$.  Thus,
        \begin{align*}
            \int_\Omega \qty(u\laplacian G - G\laplacian u)\dd \vec{\xi} &= \int_{\partial\Omega}\qty(u\frac{\partial G}{\partial n\qty(\vec{\xi})} - G\frac{\partial u}{\partial n\qty(\vec{\xi})})\dd s\qty(\vec{\xi})
        \end{align*}
        However, $u$ is harmonic in $\Omega$, i.e.~$\laplacian u \equiv 0$ for $(x,y) \in \Omega$, and $-\laplacian G = \delta\qty(\vec{x} - \vec{\xi})$, i.e.
        \begin{align*}
            \int_\Omega \qty(u\laplacian G - \cancelto{0}{G\laplacian u})\dd \vec{\xi} = -\int_\Omega u\qty(\vec{\xi})\delta\qty(\vec{x} - \vec{\xi})\dd \vec{\xi} = -u\qty(\vec{x})
        \end{align*}
        Also, the Dirichlet condition $u\qty(\vec{x}) = f(x)$ on $\partial\Omega$ implies $G \equiv 0$ for $y = 0$ (i.e.~for $(x,y) \in \partial\Omega$).  Thus,
        \begin{align*}
            -u\qty(\vec{x}) = \int_{\partial\Omega}\qty(u\frac{\partial G}{\partial n\qty(\vec{\xi})} - \cancelto{0}{G\frac{\partial u}{\partial n\qty(\vec{\xi})}})\dd s\qty(\vec{\xi}) &= \int_{\partial\Omega} u\qty(\vec{\xi})\frac{\partial G\qty(x,y;\xi,\eta)}{\partial n\qty(\vec{\xi})}\dd s\qty(\vec{\xi})
        \end{align*}
        Finally, the unit normal vector on the boundary of $\Omega$ is $-\eta$, and thus
        \begin{align*}
            \boxed{u\qty(\vec{x}) = \int_\Rl u\qty(\vec{\xi})\frac{\partial G\qty(x,y;\xi,\eta)}{\partial \eta} \dd \xi}
        \end{align*}
    \item
        \emph{Use the Green's function representation to show that $$u(x,y) = \frac{y}{\pi}\int_{-\infty}^\infty \frac{f(t)}{(x-t)^2 + y^2}\dd t.$$}

        Rewrite $G\qty(\vec{x}; \vec{\xi}) = G\qty(x,y;\xi,\eta)$.  Then
        \begin{align*}
            G\qty(\vec{x}, \vec{\xi}) &= \frac{1}{2\pi}\log\qty(\frac{\left|\vec{x} - \vec{\xi}\right|}{\left|\vec{x} - \vec{\xi^*}\right|}) = -\frac{1}{4\pi}\log\qty(\frac{(x - \xi)^2 + (y - \eta)^2}{(x - \xi)^2 + (y + \eta)^2}) \\
            \implies \left.\frac{\partial G}{\partial \eta}\right|_{\eta = 0} &= \qty[-\frac{1}{4\pi}\cdot\frac{\qty[(x - \xi)^2 + (y + \eta)^2](-2)(y - \eta) - \qty[(x - \xi)^2 + (y - \eta)^2](2)(y + \eta)}{\qty((x - \xi)^2 + (y + \eta)^2)^2}]_{\eta = 0} \\
            &= -\frac{1}{4\pi}\cdot\frac{\qty[(x - \xi)^2 + y^2](-2y) - \qty[(x - \xi)^2 + y^2](2y)}{\qty((x - \xi)^2 + y^2)^2} \\
            &= -\frac{1}{4\pi}\cdot\frac{1}{\qty(x - \xi)^2 + y^2}\cdot(-4y) \\
            &= \frac{y}{\pi}\frac{1}{\qty(x - \xi)^2 + y^2}
        \end{align*}
        Thus,
        \begin{align*}
            u(x,y) = u\qty(\vec{x}) = \int_\Rl u\qty(\vec{\xi})\frac{\partial G(x,y;\xi,\eta)}{\partial \eta}\dd \xi = \frac{y}{\pi}\int_\Rl \frac{u\qty(\vec{\xi})}{(x - \xi)^2 + y^2}\dd \xi
        \end{align*}
        Define $t \coloneqq \xi$ and note $u\qty(\vec{\xi}) = f\qty(\xi)$ when $\eta = 0$.  Thus,
        \begin{align*}
            \boxed{u(x,y) = \frac{y}{\pi}\int_\Rl \frac{f(t)}{(x-t)^2 + y^2}\dd t}
        \end{align*}
\end{enumerate}












\end{document}
