\documentclass{article} % A4 paper and 11pt font size
\setcounter{secnumdepth}{0}

\usepackage{amssymb, amsmath, amsfonts}
\usepackage{moreverb}
\usepackage{graphicx}
\usepackage{enumerate}
\usepackage{graphics}
\usepackage[margin=1.25in]{geometry}
\usepackage{color}
\usepackage{tocloft}
\renewcommand{\cftsecleader}{\cftdotfill{\cftdotsep}}
\usepackage{array}
\usepackage{float}
\usepackage{hyperref}
\usepackage{textcomp}
\usepackage[makeroom]{cancel}
\usepackage{bbold}
\usepackage{alltt}
\usepackage{physics}
\usepackage{mathtools}
\usepackage{amsthm}
\usepackage{tikz}
\usetikzlibrary{positioning}
\usetikzlibrary{arrows}
\usepackage{pgfplots}
\usepackage{bigints}
\allowdisplaybreaks
\pgfplotsset{compat=1.12}

\theoremstyle{plain}
\newtheorem*{theorem*}{Theorem}
\newtheorem{theorem}{Theorem}
\newtheorem*{lemma*}{Lemma}
\newtheorem{lemma}{Lemma}

\newenvironment{definition}[1][Definition]{\begin{trivlist}
\item[\hskip \labelsep {\bfseries #1}]}{\end{trivlist}}

\newcommand{\E}{\varepsilon}
\def\Rl{\mathbb{R}}
\def\Cx{\mathbb{C}}

\usepackage[T1]{fontenc} % Use 8-bit encoding that has 256 glyphs
\usepackage{fourier} % Use the Adobe Utopia font for the document - comment this line to return to the LaTeX default
\usepackage[english]{babel} % English language/hyphenation

\usepackage{sectsty} % Allows customizing section commands
\allsectionsfont{\centering \normalfont\scshape} % Make all sections centered, the default font and small caps

\usepackage{fancyhdr} % Custom headers and footers
\pagestyle{fancy} % Makes all pages in the document conform to the custom headers and footers
\fancyhead[L]{\bf Sam Fleischer}
\fancyhead[C]{\bf UC Davis \\ Applied Mathematics (MAT207B)} % No page header - if you want one, create it in the same way as the footers below
\fancyhead[R]{\bf Winter 2016}

\fancyfoot[L]{\bf } % Empty left footer
\fancyfoot[C]{\bf \thepage} % Empty center footer
\fancyfoot[R]{\bf } % Page numbering for right footer
\renewcommand{\headrulewidth}{0pt} % Remove header underlines
\renewcommand{\footrulewidth}{0pt} % Remove footer underlines
\setlength{\headheight}{25pt} % Customize the height of the header

\newcommand{\problem}[1]{
\vspace{.375cm}
\begin{minipage}{\textwidth}
    \begin{center}
        \noindent\rule{5cm}{1pt}
    \end{center}
    \section{\bf #1}
    \begin{center}
        \noindent\rule{5cm}{1pt}
    \end{center}
    \vspace{0.25cm}
\end{minipage}
}

\numberwithin{equation}{section} % Number equations within sections (i.e. 1.1, 1.2, 2.1, 2.2 instead of 1, 2, 3, 4)
\numberwithin{figure}{section} % Number figures within sections (i.e. 1.1, 1.2, 2.1, 2.2 instead of 1, 2, 3, 4)
\numberwithin{table}{section} % Number tables within sections (i.e. 1.1, 1.2, 2.1, 2.2 instead of 1, 2, 3, 4)

\setlength\parindent{0pt} % Removes all indentation from paragraphs - comment this line for an assignment with lots of text

\newcommand{\horrule}[1]{\rule{\linewidth}{#1}} % Create horizontal rule command with 1 argument of height

\title{ 
\normalfont \normalsize 
\textsc{UC Davis, Applied Mathematics (MAT207B), Winter 2016} \\ [25pt] % Your university, school and/or department name(s)
\horrule{2pt} \\[0.4cm] % Thin top horizontal rule
\Huge Homework \#4 \\ % The assignment title
\horrule{2pt} \\[0.5cm] % Thick bottom horizontal rule
}

\author{\huge Sam Fleischer} % Your name

\date{February 12, 2016} % Today's date or a custom date

\begin{document}\thispagestyle{empty}

\maketitle % Print the title

\makeatletter
\@starttoc{toc}
\makeatother

\pagebreak

\problem{Problem 1}
\emph{The following nonhomogeneous IBVP describes heat flow in a rod whose ends are held at temperatures $u_0$, $u_1$:}
\begin{align*}
    &u_t = u_{xx}\ \ \ \ \ \ \ \ \ 0 < x < 1,\ \ \ \ \ t > 0 \\
    &u(0, t) = u_0, \ \ \ \ \ \ \ u(1,t) = u_1 \\
    &u(x,0) = f(x)
\end{align*}
\begin{enumerate}[\bf (a)]  
    \item
        \emph{Find te steady state temperature $U(x)$ taht satisfies}
        \begin{align*}
            &U_{xx} = 0 \ \ \ \ \ 0 < x < 1 \\
            &U(0) = u_0,\ \ \ \ \ \ U(1) = u_1
        \end{align*}
    \item
        \emph{Write $u(x,t) = U(x) + v(x,t)$ and find the corresponding IBVP for $v$.  Use separation of variables to solve for $v$ and hence $u$.}
    \item
        \emph{How does $u(x,t)$ behave as $t \rightarrow \infty$?}
\end{enumerate}

\problem{Problem 2}
\emph{Define a first-order differential operator with complex coefficients acting $L^2(0, 2\pi)$ by $$A = -i \frac{\dd}{\dd x}.$$}
\begin{enumerate}[\bf (a)]
    \item
        \emph{Show that $A$ is formally self-adjoint.}
    \item
        \emph{Show that $A$ with periodic boundary conditions $u(0) = u(2\pi)$ is self-adjoint, and find the eigenvalues and eigenfunctions of the corresponding eigenvalue problem $$ -iu' = \lambda u, \qquad u(0) = u(2\pi).$$}
    \item
        \emph{What are the adjoint boundary conditions to the Dirichlet condition $u(0) = 0$ at $x = 0$?  Is $A$ with this Dirichlet boundary condition self-adjoint?  Find all eigenvalues and eigenfunctions of the corresponding eigenvalue problem $$ -iu' = \lambda u, \qquad u(0) = 0.$$  How does your result compare with the properties of finite-dimensional eigenvalue problems for matrices?}
\end{enumerate}

\problem{Problem 3}
\emph{Let $A$ be a Sturm-Liouville operator, given by $$Au = -(pu')' + qu,$$ acting in $L^2(a,b)$.  Verify that $A$ with the Robin boundary conditions $$\alpha u'(a) + u(a) = 0, \qquad u;(b) + \beta u(b) = 0$$ is self adjoint.}

\problem{Problem 4}
\emph{Show that the eigenvalues of the Sturm-Lioville problem}
\begin{align*}
    &-u'' = \lambda u \qquad 0 < x < 1 \\ &u(0) = 0, \qquad u'(1) + \beta u(1) = 0
\end{align*}
\emph{are given by $\lambda = k^2$ where $k > 0$ satisfies the equation $$ \beta \tan k + k = 0.$$  Show graphically that there is an infinite sequence of simple eigenvalues $\lambda_1 < \lambda_2 < \dots < \lambda_n < \dots$ with $\lambda_n \rightarrow \infty$ as $n \rightarrow \infty$.  What is the asymptotic behavior of $\lambda_n$ as $n \rightarrow \infty$?}

\problem{Problem 5}
\emph{THe following IBVP describes heat flow in a rod whose left end is held at temperature $0$ and whose right end loses heat to the surroundings according to Newton's law of cooling (heat flux is proportional to the temperature difference):}
\begin{align*}
    &u_t = u_{xx} \qquad 0 < x < 1,\ \ \ t > 0 \\
    &u(0, t) = 0, \qquad u'(1, t) = -\beta u(1,t) \\
    &u(x,0) = f(x)
\end{align*}
\emph{Solve this IBVP by the method of separation of variables.}

\end{document}
